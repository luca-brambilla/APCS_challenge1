\documentclass[]{article}

\usepackage{amsfonts}
\usepackage{amsmath}
\usepackage{verbatim}
\usepackage{graphicx}
\usepackage{bm}

\DeclareMathOperator*{\argmax}{arg\,max}
\DeclareMathOperator*{\argmin}{arg\,min}

%opening
\title{Advanced Programmig for Scientific Computing\\Challenge 1}
\author{Luca Brambilla}

\begin{document}
	
	\maketitle
	
	
\section{Problem}
Minimization of a multivariate function through a gradient method

$$\mathbf x = \argmin_{\mathbf y \in \mathbb R ^n} f(\mathbf y) $$

$$\mathbf x_{k+1} = \mathbf x_k - \alpha_k \nabla f(\mathbf x_k), \quad k=0,1,...,k_{max}$$

$$||\mathbf x_{k+1} - \mathbf x_k|| < \varepsilon_s$$
$$|f(\mathbf x_{k+1})-f(\mathbf x_k)| \varepsilon_r$$
$$k>k_{max}$$

$$\alpha_k = \alpha_0 \exp{-\mu k}$$
$$\alpha_k = \frac {\alpha_0}{1+\mu k}$$

\begin{equation}
	f(\mathbf x_k) - f(\mathbf x_k - \alpha_0\nabla f(\mathbf x_k)) \ge \sigma \alpha_0 ||\nabla f(\mathbf x_k)||^2
\end{equation}


Solve numerically the initially value Cauchy problem using the Crank-Nicolson numerical scheme.

$$
	\begin{cases}
		y'(t) = f(t, y(t)), \quad t \in I = (0, T] \\
		y(0) = y_0
	\end{cases}
$$
Considering in particular the problem:
$$
\begin{cases}
	y'(t) = -t e^{-y} , \quad t \in I = (0, T] \\
	y(0) = 0
\end{cases}
$$


\section{Solution}

\subsection{Analytic solution}
Analytic solution of the nonlinear $1^{st}$ order ODE using separation of variables

$$\int e^y dy = \int -t dt \Longleftrightarrow e^y = -\frac{1}{2} x^2 + C \Longleftrightarrow y = \log \left( C - \frac{1}{2} x^2 \right)$$
Finally, using the initial condition $y(0)=0$ we get the exact solution:
$$y_{ex} (t) = \log\left( 1 - \frac{1}{2} x^2 \right)$$

\newpage

\subsection{Numerical approximation}
Now considering the sumerical approximation of the IV problem using Crank-Nicolson numerical scheme. 

Being $N$ a positive integer and $h = T /N$ the time step, the method consists in finding $u_n \simeq y(t_n )$ for $t_n = nh$.
At each iteration we have:
$$u_{n+1} = u_n + \frac{h}{2} \left( f(t_{n+1},u_{n+1}) + f(t_{n},u_{n})  \right) \qquad \forall n = 0, \ldots , N - 1, \; u_0 = y_0 . $$

The method is implicit in $u_{n+1}$, therefore at each iteration we must find the zero of:
\begin{equation}\label{eq::nonlin}
	F(x) = x - u_n - \frac{h}{2} \left( f(t_{n+1}, x) + f(t_{n},u_{n})  \right) 
\end{equation}
given we know $u_n$ from the previous step.

Adopting the Newton-Rapson method to find the zero of each iteration $n$:
$$x_{k+1} = x_{k} - \frac{F(x_k)}{F'(x_k)}$$
using the derivative of $F(x)$: 
\begin{equation}\label{eq::nonlinder}
	F'(x) = 1 - \frac{h}{2}  f_x(t_{n+1}, x)
\end{equation}
The initial guess is set equal to the approximated solution from the previous step $x_0 = u_n$. If the residual  and $k$ satisfy the stopping criteria:
$$|x_{k+1} - x_{k} | =\left|-  \frac{F(x_k)}{F'(x_k)} \right|  < TOL, \quad k<k_{max}$$
we set $u_{n+1} = x_{k+1}$ after $k$ iterations.

In this particular case we will consider:
$$f(t, y(t)) = -t e^{-y}, \qquad f_y (t, y(t)) = +t e^{-y} $$
to be plugged as part of $F(x)$ and $F'(x)$. In Figure \ref{fig::solution} we can see the numerical solution of the problem compared with the exact solution.

\begin{figure}[h!]
	\centering
	%\includegraphics[width=0.8\textwidth]{}
	\caption{Plot of the solution of the problem. Comparing the exact solution and the approximated solution.}
	\label{fig::solution}
\end{figure}

\newpage

\subsection{Convergence analysis}

We find quadratic convergence as expected:
$$| |  u - u_h ||_{L^\infty( I )} \leq Ch^2$$
We can see it in Figure \ref{fig::order2}.

\begin{figure}[t!]
	\centering
	%\includegraphics[width=0.8\textwidth]{../figures/conv_theta0.500000.pdf}
	\caption{The convergence analysis shows a $2^{nd}$ order convergence of the solution with respect to the discretization.}
	\label{fig::order2}
\end{figure}


\section{Extensions}

\subsection{$\theta$-method}

Extension to more general $\theta$ method with $\theta \in [0,1]$:
$$u_{n+1} = u_n + h \left( (1-\theta) f(t_{n+1},u_{n+1}) + \theta f(t_{n},u_{n})  \right) $$

\begin{enumerate}
\item $\theta = 0$ to retrieve forward Euler method.
\item $\theta = 1$ to retrieve backward Euler method.
\item $\theta = \frac{1}{2}$ to retrieve Crank-Nicolson.
\end{enumerate}
In this more general case Equations \ref{eq::nonlin} and \ref{eq::nonlinder} become:
$$F(x) = x - u_n - h \left( (1-\theta) f(t_{n+1}, x) + \theta f(t_{n},u_{n})  \right) $$ 
$$F'(x) = 1 - h (1-\theta) f_x(t_{n+1}, x)  $$ 

Changing the value oh $\theta$ the order of convergence changes and we get linear convergence, as expected (see Figure \ref{fig::order1}).

\begin{figure}[h!]
	\centering
	%\includegraphics[trim={1.5cm 0 2cm 1cm},clip, width=0.49\textwidth]{../figures/conv_theta0.000000.pdf} \hfill
	%\includegraphics[trim={1.5cm 0 2cm 1cm},clip, width=0.49\textwidth]{../figures/conv_theta1.000000.pdf}
	\caption{The convergence analysis shows a $1^{st}$ order convergence of the solution with respect to the discretization.}
	\label{fig::order1}
\end{figure}

\newpage

\subsection{\texttt{gnuplot}, \texttt{GetPot}, \texttt{muparser}}

Imported header files to add more functionality:
\begin{enumerate}
	\item  Imported \texttt{gnuplot\_iostream} to use \texttt{gnuplot} directly from the source code. 
	\item Imported \texttt{GetPot} to read from an external text  file. 
	\item Imported \texttt{muparser} to use an arbritrary function from a file. 
\end{enumerate}


\section{Problems along the way}

\begin{enumerate}
	\item I initially wanted to implement a class, but I was not able to pass a non-static member function as a parameter of another function.
	Memory is not yet allocated for the class object to call a member. I read to use \texttt{std::bind} or \texttt{std::invoke} but I could not figure out how.
	\item \texttt{gnuplot} locale was not set properly and I had to look how to make it.
	\begin{verbatim}
		warning: iconv failed to convert degree sign
	\end{verbatim}
	\item  Every time I plot something I get the warning:
	{
	\small
	\begin{verbatim}
		QSocketNotifier: Can only be used with threads started with QThread
	\end{verbatim}
	}
	I am using GNOME on Fedora 37.
	\item I had issues with \texttt{muparser} when trying to create a self-contained code.
	I tried to copy all shared objects file from the \texttt{pacs-examples/Examples/lib} directory, but I was not able to copy the symbolic links.
	I got the following error from the linker:
	\begin{verbatim}
		cannot find -lmuparser
	\end{verbatim}
\end{enumerate}

\end{document}
